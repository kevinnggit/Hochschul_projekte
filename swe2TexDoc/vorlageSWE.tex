% Options for packages loaded elsewhere
\PassOptionsToPackage{unicode}{hyperref}
\PassOptionsToPackage{hyphens}{url}
%
\documentclass[]{article}
\usepackage{amsmath,amssymb}
\usepackage{iftex}
\usepackage{color} 
\ifPDFTeX
  \usepackage[T1]{fontenc}
  \usepackage[utf8]{inputenc}
  \usepackage{textcomp} % provide euro and other symbols
\else % if luatex or xetex
  \usepackage{unicode-math} % this also loads fontspec
  \defaultfontfeatures{Scale=MatchLowercase}
  \defaultfontfeatures[\rmfamily]{Ligatures=TeX,Scale=1}
\fi
\usepackage{lmodern}
\ifPDFTeX\else
  % xetex/luatex font selection
\fi
% Use upquote if available, for straight quotes in verbatim environments
\IfFileExists{upquote.sty}{\usepackage{upquote}}{}
\IfFileExists{microtype.sty}{% use microtype if available
  \usepackage[]{microtype}
  \UseMicrotypeSet[protrusion]{basicmath} % disable protrusion for tt fonts
}{}
\usepackage{xcolor}
\usepackage{longtable,booktabs,array}
\usepackage{calc} % for calculating minipage widths
% Correct order of tables after \paragraph or \subparagraph
\usepackage{etoolbox}
% Allow footnotes in longtable head/foot
\IfFileExists{footnotehyper.sty}{\usepackage{footnotehyper}}{\usepackage{footnote}}
\makesavenoteenv{longtable}
\setlength{\emergencystretch}{3em} % prevent overfull lines
\providecommand{\tightlist}{%
  \setlength{\itemsep}{0pt}\setlength{\parskip}{0pt}}
\ifLuaTeX
  \usepackage{selnolig}  % disable illegal ligatures
\fi
\usepackage{bookmark}
\usepackage[ngerman]{babel}
\IfFileExists{xurl.sty}{\usepackage{xurl}}{} % add URL line breaks if available
\urlstyle{same}
\usepackage[a4paper, total={6in, 8in}]{geometry}

\usepackage{scrlayer-scrpage}
\pagestyle{scrheadings}
\clearpairofpagestyles
\ihead{Projektname}
\ohead{\today}
\chead{Spezifikation der Softwarevoraussetzungen Version}
\ifoot{Teamname}
\ofoot{Seite \thepage}
\cfoot{\textsuperscript{\textcopyright} Unternehmen 2024}

\pagenumbering{arabic}
\author{}
\date{}

\begin{document}

\title{
\begin{flushright}
\textless Projektname\textgreater \\
Spezifikation der Softwarevoraussetzungen\footnote{ 
  \textsuperscript{} \emph{Dieses Template wird anstelle eines
  Pflichtenheftes verwendet!} 
  }\\
für \textless System \textgreater \\
Version \textless x.y\textgreater{}
\end{flushright}
}


\maketitle

\color{blue}
{[}Anmerkung: Die folgende Vorlage\footnote{\textsuperscript{}
  \emph{Angepasstes Template auf Basis der Templates für ``Software
  Requirements Specification'' und ''Vision'' }

  \emph{des IBM Rational Unified Process}} basiert auf einer Vorlage von
IBM. Blau dargestellte Texte in eckigen Klammern sind
Anleitungen zum Ausfüllen des Dokumentes und sollten vor der
Veröffentlichung des Dokuments gelöscht werden.
 
\color{black}

\setcounter{secnumdepth}{0} % Keine Nummerierung der Section
\section{Revisionsprotokoll}

\setcounter{secnumdepth}{3} % Folgende Sections werden wieder nummiert

\begin{longtable}[]{|l|l|l|l|}
\hline
\textbf{Änderungsdatum} & \textbf{Version } & \textbf{Beschreibung} &
\textbf{Autor} \\
\hline
\textless tt/mm/jj\textgreater{} & \textless x.y\textgreater{} &
\textless Details\textgreater{} & \textless Name\textgreater{} \\
\hline
& & & \\
\hline
& & & \\
\hline
& & & \\
\hline
\end{longtable}
\clearpage

\tableofcontents
\newpage

\section{Einführung}

\color{blue}
{[}Die Einführung zur \textbf{Spezifikation der Softwareanforderungen
}gibt einen Überblick über die gesamte \textbf{Spezifikation}. Die
Einführung gibt außerdem Verwendungszweck, Umfang, Definitionen,
Akronyme, Abkürzungen und Referenzen der \textbf{Spezifikation der
Softwareanforderungen }an.{]}


{[}Anmerkung: Diese Spezifikation deckt sämtliche
Softwareanforderungen für das System oder einen Teil des
Systems ab.{]}


{[}\textbf{Spezifikationen der Softwareanforderungen} können ganz
unterschiedlich aufgebaut sein. Wenn Sie sich über die hier gegebenen
Erläuterungen hinaus mit diesem Thema beschäftigen möchten und sich für
weitere Optionen beim Aufbau von \textbf{Spezifikationen der
Softwareanforderungen }interessieren, sollten Sie {[}IEEE830-1998{]}
lesen.{]}

\color{black}
\subsection{Verwendungszweck}

\color{blue}
{[}Geben Sie den Verwendungszweck dieses Dokuments an.{]}

\color{black}
\subsection{Definitionen, Akronyme und Abkürzungen}
\color{blue}
{[}Dieser Unterabschnitt enthält eine Definition aller Begriffe,
Akronyme und Abkürzungen, damit die \textbf{Spezifikation} ordnungsgemäß
interpretiert werden kann.{]}(Glossar)

\color{black}
  \section{Allgemeine Beschreibung}
	
\color{blue}
{[}In diesem Abschnitt sind die allgemeinen Faktoren beschrieben, die
das Produkt und die geltenden Anforderungen beeinflussen. Dieser
Abschnitt enthält keine spezifischen Anforderungen. Er gibt vielmehr
Hintergrundinformationen zu den Anforderungen, die im Abschnitt 3
detailliert aufgeführt sind, um sie besser verständlich zu machen.
Nehmen Sie Punkte wie die folgenden auf:
\color{black}

\subsection{Geschäftschance}

\color{blue}
{[}Beschreiben Sie kurz die Geschäftschance, die dieses Projekt
bietet.{]}
\color{black}

\subsection{Problembeschreibung}

\color{blue}
{[}Beschreiben Sie das Problem, das durch dieses Projekt gelöst werden
soll. Sie können das folgende Format verwenden:{]}

\color{black}
\begin{longtable}[]{|p{\columnwidth * \real{0.5}}| p{\columnwidth * \real{0.5}}|} \hline
 
Das Problem, 
  &  
\color{blue} {[}Problem beschreiben{]}
  \\ \hline
 
\color{black}betrifft
  &  
\color{blue}{[}die vom Problem betroffenen Stakeholder{]}
  \\ \hline
 
\color{black}Auswirkung des Problems:
  &  
\color{blue}{[}Was sind die Auswirkungen des Problems?{]}
  \\ \hline
 
\color{black}Eine erfolgreiche Lösung besteht darin,
  &  
\color{blue}{[}Einige wichtige Vorzüge einer erfolgreichen Lösung auflisten{]}
  \\ \hline
\end{longtable}

\color{black}

   \subsection{Produktpositionierung}
	
\color{blue}
{[}Allgemeine Aussage, die auf der höchsten Ebene zusammenfassend
beschreibt, welche einzigartige Position das Produkt am Markt einnehmen
soll. Das folgende Format kann verwendet werden:{]}

\color{black}
\begin{longtable}[\textwidth]{| p{\columnwidth * \real{0.5}}| p{\columnwidth * \real{0.5}}|}\hline

Für
 & 
\color{blue}{[}Zielgruppen{]},
 \\ \hline
\color{black} \vtop{\hbox{\strut erfüllt das Produkt /}\hbox{\strut ermöglicht das
Produkt}}
 & 
\color{blue} {[}Aussage zum Bedarf bzw. zur Gelegenheit{]}
 \\
\color{black} Das vorliegende Produkt (Produktname)
 & 
\color{blue} ist ein(e) {[}Produktkategorie{]}
 \\ \hline
\color{black} und zeichnet sich dadurch aus, dass
 & 
\color{blue}{[}Aussage zum entscheidenden Vorteil des vorliegenden Produkts (zwingender Kaufgrund){]} \\ \hline
\color{black} Im Gegensatz zu & 
\color{blue}{[}primäre Konkurrenzprodukte{]} \\ \hline
\color{black} hat das vorliegende Produkt folgende Eigenschaften
 & 
\color{blue}{[}Aussage zur grundlegenden Abgrenzung von Konkurrenzprodukten{]} \\ \hline
\end{longtable}

{[}Eine Produktpositionierung kommuniziert allen beteiligten
Mitarbeitern den Zweck der Anwendung und die Bedeutung des Projekts.{]}

\color{black}

 \subsection{Benutzerübersicht}

\color{blue}
{[}Erstellen Sie eine zusammenfassende Liste aller identifizierten
Benutzer.{]}
\color{black}

\begin{longtable}[]{|
 p{\columnwidth * \real{0.33}}|
 p{\columnwidth * \real{0.33}}|
 p{\columnwidth * \real{0.33}}|}\hline

& & \\ 

\textbf{Name} & 
\textbf{Beschreibung}
  & 
\textbf{Zuständigkeiten}
 \\ \hline
\color{blue}
{[}Geben Sie den Benutzertyp an.{]}
 & 
\color{blue}{[}Beschreiben Sie kurz, welche Funktion die Benutzer im Hinblick auf
das System haben.{]}
 & 
\color{blue}{[}Listen Sie die wichtigsten Zuständigkeiten des Benutzers im Hinblick
auf das zu entwickelnde System auf. Beispiele:


\begin{itemize}
\item Der Benutzer erfasst Einzeldaten.
\item Der Benutzer erstellt Berichte.
\item Der Benutzer koordiniert die Arbeit.
\item usw.{]}
  
\end{itemize} \\ \hline
\end{longtable}

\color{black}

  \subsection[Benutzerumgebung]{Benutzerumgebung}

\color{blue}
{[}Geben Sie die Arbeitsumgebung des Zielbenutzers detailliert an.
Einige Vorschläge zur Vorgehensweise:

Wie viele Personen sind an der Ausführung der Aufgabe beteiligt? Ändert
sich die Anzahl?

Wie lang ist ein Aufgabenzyklus? Zeitraum pro Aktivität? Ändert sich
das?

Gibt es eindeutige Umweltbedingungen: mobil, draußen, an Bord?

Welche Systemplattformen werden heute verwendet? Welche Plattformen
sollen in Zukunft verwendet werden?

Welche anderen Anwendungen werden verwendet? Muss Ihre Anwendung in
diese Anwendungen integriert werden?

Hier können Auszüge aus dem Geschäftsmodell aufgenommen werden, um die
entsprechenden Aufgaben, Rollen usw. zu umreißen.{]}\\

\color{black}

    \subsection{Bedarf und Features}
\color{blue}

{[}Vermeiden Sie technische Details. Beschreiben Sie die Features nur
allgemein. Konzentrieren Sie sich auf die erforderlichen Funktionen, und
geben Sie an, warum (nicht wie) sie implementiert werden sollen.{]}

\color{black}
\begin{longtable}[]{| p{\columnwidth * \real{0.3333}}| p{\columnwidth * \real{0.3333}}| p{\columnwidth * \real{0.3333}}|} \hline

\textbf{Bedarf}
 & 
\textbf{Priorität}
 & 
\textbf{Features}
 \\ \hline
 &  &  \\ \hline
 &  &  \\ \hline
 &  & \\ \hline
 & &  \\ \hline
 &  &  \\ \hline
\end{longtable}


\subsection{Produktperspektive}

\color{blue}

{[}Dieser Unterabschnitt der \textbf{Spezifikation der
Softwareanforderungen} setzt das Produkt in Relation zu vergleichbaren
anderen Produkten und zur Umgebung des Benutzers. Falls es sich um ein
vollkommen unabhängiges Produkt handelt, geben Sie dies hier an. Sollte
das Produkt eine Komponente eines größeren Systems sein, müssen Sie in
diesem Unterabschnitt angeben, wie die Systeme interagieren und welche
Schnittstellen es zwischen den Systemen gibt. Am leichtesten lassen sich
die Hauptkomponenten eines größeren Systems sowie deren Wechselwirkungen
und externe Schnittstellen in einem Diagramm (z.B. einem
\textbf{UML-Paketdiagramm}) darstellen.{]}

\color{black}

    \subsection{Annahmen und Abhängigkeiten}
\color{blue}
{[}Dieser Abschnitt beschreibt alle wichtigen Annahmen zur technischen
Durchführbarkeit, zur Verfügbarkeit von Subsystemen oder Komponenten
sowie weitere projektbezogene Annahmen, von denen die Wirtschaftlichkeit
der Software, die in dieser \textbf{Spezifikation der
Softwareanforderungen} beschrieben ist, abhängen kann.{]}

\color{black}

  \section{Spezifische Anforderungen (Funktionale Anforderungen)}
	
\color{blue}	

{[}In diesem Abschnitt der \textbf{Spezifikation der
Softwareanforderungen} sind alle Softwareanforderungen so detailliert
angegeben, dass Entwickler ein System entwerfen können, das diese
Anforderungen erfüllt, und dass Tester überprüfen können, ob das System
den genannten Anforderungen gerecht wird. Wenn Sie mit
Anwendungsfallmodellen arbeiten, werden diese Anforderungen in
Anwendungsfällen und den geltenden ergänzenden Spezifikationen erfasst.
Das ist in SWE II der Fall.{]}

\color{black}

    \subsection{UML-Anwendungsfalldiagramm}
		
\color{blue}
{[}Fügen Sie hier Ihr/e Anwendungsfalldiagramm/e ein. (Screenshot){]}\\
\color{black}

    \subsection{Anwendungsfall-Spezifikationen in Form von UML-Aktivitätsdiagrammen}
		
\color{blue}		
{[}Wenn Anwendungsfallmodelle erstellt werden, definieren die
Anwendungsfälle häufig die meisten funktionalen Anforderungen des
Systems und einige nicht funktionale Anforderungen.\\
Nehmen Sie in diesen Abschnitt die Anwendungsfall-Spezifikation zu jedem
Anwendungsfall im obigen Anwendungsfalldiagramm oder zu ausgewählten
Anwendungsfällen des Modells auf oder fügen Sie hier Verweise auf diese
Spezifikationen ein. Stellen Sie sicher, dass jede Anforderung klar
benannt ist.{]}\\
(Fügen Sie in den Unterkapiteln die zum jeweiligen Anwendungsfall
gehörigen Anwendungsfall-Spezifikation in Form von Aktivitätsdiagrammen
mit Schwimmbahnen sowie Vor- und Nachbedingungen ein.)

\color{black}

      \subsubsection{UML-Aktivitätsdiagramm zu \textless Name des Anwendungsfalles a\textgreater{}}
\emph{(Screenshot des zugehörigen Aktivitätsdiagrammes)}

      \subsubsection{UML-Aktivitätsdiagramm zu \textless Name des Anwendungsfalles b\textgreater{}}

.....
  \section{Ergänzende Anforderungen (Nicht-funktionale Anforderungen)}
\color{blue}

{[}Anforderungen, die nicht in den Anwendungsfällen enthalten sind,
werden in ergänzenden Spezifikationen dokumentiert. Die speziellen
Anforderungen aus diesen Spezifikationen, die für dieses Subsystem oder
Feature gelten, sollten hier eingefügt und so weit präzisiert werden,
dass sie das Subsystem oder Feature detailliert beschreiben. Sie können
an dieser Stelle aber auch auf die einzelnen ergänzenden Spezifikationen
verweisen. Stellen Sie sicher, dass jede Anforderung klar benannt
ist.{]}\\
Wenn es zu einem Abschnitt keine relevanten Anforderungen gibt, löschen
Sie diesen.

\color{black}

    \subsection{Benutzungsfreundlichkeit }

\color{blue}
{[}Dieser Abschnitt enthält alle Anforderungen mit Einfluss auf die
Benutzungsfreundlichkeit. Beispiele:
\begin{itemize}
	\item   Geben Sie die Schulungszeit an, die erforderlich ist, bis normale und
  professionelle Benutzer bestimmte Operationen produktiv ausführen
  können.
	\item   Geben Sie messbare Zeiten für typische Aufgaben an oder verwenden Sie
  die Benutzerfreundlichkeit bekannter Systeme, die bei den Benutzern
  beliebt sind, als Grundlage für die Benutzerfreundlichkeit des neuen
  Systems.
	\item 
  Nennen Sie Anforderungen an die Konformität mit allgemeinen Standards
  für die Benutzerfreundlichkeit an, z. B. mit den CUA-Standards der IBM
  oder mit den GUI-Standards von Microsoft.{]}
\end{itemize}


\color{black}
    \subsection{Zuverlässigkeit}
		
\color{blue}
{[}Hier sollten Sie Anforderungen an die Zuverlässigkeit des Systems
angeben. Vorschläge:

\begin{itemize}
\item
  Verfügbarkeit ---Geben Sie die Verfügbarkeitszeit in Prozent an (xx,xx
  \%), die Nutzungsdauer, den Wartungszugriff, den Betrieb im
  eingeschränkten Modus und dergleichen mehr.
\item
  Mittlere Zeit zwischen auftretenden Fehlern --- Diese Zeit wird
  normalerweise in Stunden angegeben, kann aber auch in Tagen, Monaten
  oder Jahren angegeben werden.
\item
  Mittlere Reparaturzeit ---Wie lange kann das System bei einem Ausfall
  außer Betrieb bleiben?
\item
  Genauigkeit---Geben Sie die für Systemausgaben erforderliche
  Genauigkeit (Auflösung) und Präzision (nach bekanntem Standard) an.
\item
  Maximale Programmfehler- oder Mängelrate --- Diese Rate wird in der
  Regel in Programmfehlern pro tausend Codezeilen oder in
  Programmfehlern pro Funktionspunkt angegeben.
\item
  Programmfehler- oder Mängelquote --- Diese Quote wird als Anzahl
  geringfügiger, signifikanter und kritischer Programmfehler angegeben.
  Die Anforderungen müssen definieren, was unter einem ``kritischen''
  Programmfehler zu verstehen ist (beispielsweise ein vollständiger
  Datenverlust oder die Unmöglichkeit, bestimmte Funktionen des Systems
  zu verwenden.{]}
\end{itemize}
\color{black}

    \subsection{Leistung}
		
\color{blue}
{[}Umreißen Sie in diesem Abschnitt die Leistungsmerkmale des Systems.
Geben Sie spezifische Antwortzeiten an.

\begin{itemize}
  \item Antwortzeit für eine Transaktion (durchschnittlich, maximal)

  \item Durchsatz (z. B. Transaktionen pro Sekunde)

  \item Kapazität (z. B. mögliche Anzahl von Kunden oder Transaktionen)

  \item Eingeschränkte Modi (Welcher Betriebsmodus ist akzeptabel, wenn das
  System in irgendeiner Form eingeschränkt ist?)

  \item Ressourcennutzung (Speicher, Datenträger, Kommunikation usw.){]}
\end{itemize}

\color{black}
    \subsection{Servicefreundlichkeit}
\color{blue}

{[}Dieser Abschnitt enthält alle Anforderungen, die der Verbesserung der
Servicefreundlichkeit oder Wartungsfreundlichkeit des zu erstellenden
Systems dienen. Dazu gehören Codierungsstandards, Namenskonventionen,
Klassenbibliotheken, Wartungszugriff und Wartungsdienstprogramme.{]}
\color{black}

    \subsection{Designeinschränkungen}
\color{blue}

{[}In diesem Abschnitt werden alle Designeinschränkungen des zu
erstellenden Systems angegeben. Designeinschränkungen sind
Designentscheidungen, die obligatorisch und zu respektieren sind.
Beispiele sind Einschränkungen für Softwaresprachen, Anforderungen an
Softwareprozesse, Vorschriften für die Verwendung von Entwicklungstools,
Rahmenbedingungen für Architektur und Design, gekaufte Komponenten,
Klassenbibliotheken usw.{]}

\color{black}
    \subsection{Anforderungen an die Onlinedokumentation für Benutzer und das Hilfesystem}
\color{blue}

{[}Beschreiben Sie hier die Anforderungen an die Onlinedokumentation für
Benutzer, das Hilfesystem, Hilfe zu Anmerkungen usw.{]}
\color{black}

    \subsection{Gekaufte Komponenten}
\color{blue}

{[}Dieser Abschnitt beschreibt alle Komponenten, die für das System
gekauft werden, alle geltenden Lizenzierungs- oder Nutzungsbedingungen
sowie alle zugehörigen Kompatibilitäts- und Interoperabilitätsstandards
oder Schnittstellenstandards.{]}
\color{black}

    \subsection{Schnittstellen}
\color{blue}

{[}Dieser Abschnitt definiert die Schnittstellen, die die Anwendung
unterstützen muss. Er sollte mit angemessener Genauigkeit Protokolle,
Ports und logische Adresse etc. angeben, so dass die Software auf der
Basis der Schnittstellenanforderungen entwickelt und geprüft werden
kann.{]}
\color{black}

      \subsubsection{Benutzerschnittstellen}
\color{blue}

{[}Beschreiben Sie die Benutzerschnittstellen, die von der Software
implementiert werden müssen.{]}
\color{black}

      \subsubsection{Hardwareschnittstellen}
\color{blue}
{[}Dieser Abschnitt definiert alle Hardwareschnittstellen
(einschließlich der logischen Struktur, der physischen Adressen, des
erwarteten Verhaltens usw.), die die Software unterstützen muss.{]}
\color{black}

      \subsubsection{Softwareschnittstellen}
\color{blue}
{[}Dieser Abschnitt beschreibt Softwareschnittstellen zu anderen
Komponenten des Softwaresystems. Es kann sich um gekaufte Komponenten
handeln, um wiederverwendete Komponenten einer anderen Anwendung oder um
Komponenten, die für ganz andere Subsysteme entwickelt wurden, mit denen
diese Softwareanwendung jedoch interagieren muss.
\color{black}

      \subsubsection{Übertragungsschnittstellen}
\color{blue}

{[}Beschreiben Sie alle Übertragungsschnittstellen zu anderen Systemen
oder Einheiten (z. B. lokale Netze, ferne serielle Einheiten etc.).{]}
\color{black}

    \subsection{Lizenzanforderungen}
\color{blue}

{[}Dieser Abschnitt definiert alle Anforderungen an die Lizenzierung
oder an andere Nutzungsbeschränkungen für die Software.{]}
\color{black}

    \subsection{Rechtliche Hinweise, Copyright und Bemerkungen}
\color{blue}

{[}Dieser Abschnitt beschreibt den rechtlich zulässigen
Haftungsausschluss, die erforderliche Gewährleistung,
Copyright-Vermerke, bestehende Patente, geschützte Wörter, Marken oder
Logos und Konformitätsaspekte der Software.{]}
\color{black}

    \subsection{Geltende Standards}

\color{blue}
{[}Dieser Abschnitt verweist auf alle geltenden Standards und gibt die
spezifischen Abschnitte des jeweiligen Standards an, die auf das zu
beschreibende System anwendbar sind. Dabei kann es sich um gesetzliche
Vorschriften, Qualitätsnormen und Verwaltungsvorschriften,
Industrienormen für Benutzerfreundlichkeit, Interoperabilität,
Internationalisierung, Betriebssystemkonformität usw. handeln.{]}

\color{black}
\setcounter{secnumdepth}{0}
\section[Anhang]{Anhang}

\subsection[A Klassendiagramm]{A Klassendiagramm}

\subsection[B ggf. UML-Sequenz- und Zustandsdiagramme]{B ggf. UML-Sequenz- und Zustandsdiagramme}

\subsection[C Personas]{C Personas}

\subsection[D Storyboard]{D Storyboard}

\subsection[E Link auf den Online-Prototypen und Screenshots des Prototypen]{E Link auf den Online-Prototypen und Screenshots des Prototypen}

\end{document}
